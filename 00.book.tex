%%%%%%%%%%%%%%%%%%%% book.tex %%%%%%%%%%%%%%%%%%%%%%%%%%%%%
%
% sample root file for the chapters of your "monograph"
%
% Use this file as a template for your own input.
%
%%%%%%%%%%%%%%%% Springer-Verlag %%%%%%%%%%%%%%%%%%%%%%%%%%


% RECOMMENDED %%%%%%%%%%%%%%%%%%%%%%%%%%%%%%%%%%%%%%%%%%%%%%%%%%%
\documentclass[pdftex,12pt, oneside]{article}

% choose options for [] as required from the list
% in the Reference Guide, Sect. 2.2
%\usepackage[paperwidth=8.5in, paperheight=13in]{geometry} % Folio
\usepackage[paperwidth=8.27in, paperheight=11.69in]{geometry} % A4

\usepackage{makeidx}         % allows index generation
\usepackage{graphicx}        % standard LaTeX graphics tool
                             % when including figure files
%\usepackage{multicol}        % used for the two-column index
\usepackage[bottom]{footmisc}% places footnotes at page bottom
\usepackage[english]{babel}
\usepackage{enumerate}
\usepackage{paralist}
\usepackage{float}
\usepackage{gensymb}  
\usepackage{listings}
\usepackage{mathtools} % atau \usepackage{amsmath}
%\usepackage{siunitx}
% etc.
% see the list of further useful packages
% in the Reference Guide, Sects. 2.3, 3.1-3.3
\renewcommand{\baselinestretch}{1.5}

\newcommand{\HRule}{\rule{\linewidth}{0.5mm}}

%\makeindex             % used for the subject index
                       % please use the style svind.ist with
                       % your makeindex program


%%%%%%%%%%%%%%%%%%%%%%%%%%%%%%%%%%%%%%%%%%%%%%%%%%%%%%%%%%%%%%%%%%%%%

\begin{document}
\sloppy % biar section ga melebar melewati kertas
%\author{Priyanto Tamami}
%\title{BUKU PETUNJUK OPERASIONAL SISTEM INFORMASI GEOGRAFIS UNTUK PBB-P2 DENGAN MAPINFO VERSI 8.0}
%\date{22 Desember 2015}
%\maketitle

%\input{./01.title.tex}
\begin{center}
{\large STUDI KELAYAKAN RINCI PENGOLAHAN DATA SISTEM BEA PEROLEHAN HAK ATAS TANAH DAN/ATAU BANGUNAN}
\\[1cm]
16 Maret 2016\\
Priyanto Tamami, S.Kom.
\end{center}

%\frontmatter%%%%%%%%%%%%%%%%%%%%%%%%%%%%%%%%%%%%%%%%%%%%%%%%%%%%%%

%\include{dedic}
%\include{pref}

%\include{02.pengesahan} 

%\tableofcontents
%\listoffigures

%\mainmatter%%%%%%%%%%%%%%%%%%%%%%%%%%%%%%%%%%%%%%%%%%%%%%%%%%%%%%%
%\include{part}
%\include{chapter}
%\include{chapter}
%\appendix
%\include{appendix}

%\include{03.konsep-sig}
%\include{04.pengenalan-software}
%\include{05.koordinat}
%\include{06.registrasi-transformasi-koordinat} 
%\include{07.digitasi-on-screen} 
%\include{08.query} 

%\backmatter%%%%%%%%%%%%%%%%%%%%%%%%%%%%%%%%%%%%%%%%%%%%%%%%%%%%%%%
%\include{solutions}
%\include{referenc}
%\printindex

%%%%%%%%%%%%%%%%%%%%%%%%%%%%%%%%%%%%%%%%%%%%%%%%%%%%%%%%%%%%%%%%%%%%%%

\section{RUANG LINGKUP PEKERJAAN}

Karena luasnya ruang lingkup studi kelayakan tentang pengembangan sistem informasi Bea Perolehan Hak atas Tanah dan/atau Bangunan (BPHTB), dan dengan alasan bahwa dibangunnya sistem informasi sebagai salah satu kebutuhan internal baik sebagai alat untuk pencatatan administrasi yang lebih mudah dan aman, tentunya berkaitan dengan proses pemasukkan data yang mengurangi kesalahan pada sisi \textit{human error}, selain itu dapat pula dilakukan pengembangan-pengembangan fasilitas dari sistem informasi sesuai dengan kebutuhan akan data-data dan informasi yang berkaitan dengan pengelolaan BPHTB, maka beberapa kegiatan yang nantinya dilakukan dalam studi kelayakan ini yaitu :

\begin{enumerate}[1.]
\item Wawancara

Wawancara dilakukan dengan personal yang menangani langsung kegiatan pencatatan administrasi BPHTB, ini dilakukan karena sistem informasi yang akan dibangun nantinya diharapkan dapat mempermudah pekerjaan seorang \textit{data entry} atau personil yang berada di pelayanan dalam mencatatkan berkas pengajuan pelayanan BPHTB. Hal teknis sekecil apapun akan sangat menentukan bagaimana sistem informasi BPHTB nantinya dibangun, baik dalam bentuk data apa saja yang perlu disimpan dalam basis data, tentunya ini akan menentukan bagaimana bentuk tampilan dari aplikasi yang nantinya akan digunakan.

\item Pengumpulan Bahan-Bahan

Pengumpulan bahan-bahan hasil produksi pencatatan yang dilakukan secara manual diperlukan karena ini yang nantinya jadi hasil keluaran bagi sistem informasi BPHTB, sehingga nantinya dapat ditentukan bagaimana sistem informasi akan bekerja untuk mengumpulkan data yang telah dimasukkan dan pada akhirnya akan menghasilkan keluaran yang tidak jauh berbeda dengan apa yang biasa diproduksi pada saat dilakukannya pencatatan administrasi secara manual.
\end{enumerate}

\section{SARANA DAN PRASARANA YANG MELIPUTI PERANGKAT KERAS DAN PERANGKAT LUNAK YANG DIPERLUKAN}

Kebutuhan akan sarana dan prasarana untuk jenis perangkat keras yaitu :

\begin{enumerate}[1.]
  \item Server Basis Data
  
Server ini berfungsi untuk menyimpan basis data hasil \textit{entry}, untuk nantinya dapat diakses oleh server aplikasi web (\textit{web app server}) yang dapat dikembangkan bukan hanya sebagai alat untuk \textit{entry} data BPHTB, melainkan dapat menampilkan informasi-informasi lain seperti statistik pelayanan, ataupun kondisi pelaporan lain.

Kebutuhan perangkat keras server ini tidak perlu membeli karena nantinya akan menggunakan server yang sudah ada sebagai server virtualisasi. Karena \textit{resource} simpanan data pada server virtualisasi masih cukup luas dan penggunaannya hanya dilakukan pada saat cetak massal Surat Pemberitahuan Pajak Terhutang (SPPT) Pajak Bumi dan Bangunan Perdesaan dan Perkotaan (PBB-P2).

Jadi server vertualisasi ini akan memiliki beberapa personal komputer virtual didalamnya yang memiliki alamat IP (\textit{Internet Protocol}) mandiri yang dapat diakses dan mengakses jaringan komputer, karena bentuknya adalah personal komputer, maka disana dapat dipasang (\textit{di-install}) aplikasi SISMIOP yang kemudian tentu saja personal komputer virtual ini akan mengakses \textit{printer} untuk melakukan cetak SPPT PBB-P2 di awal tahun.

  \item Server Web Aplikasi
  
Server ini akan digunakan sebagai tempatnya aplikasi web, jadi setiap personal komputer yang akan melakukan \textit{entry} data, atau mengakses informasi dari data-data yang tersimpan di server basis data akan membuka aplikasi yang disediakan oleh \textit{server} ini.
  
Server ini pun tidak perlu membeli karena nantinya akan menggunakan server rekonsiliasi penerimaan PBB-P2. Alasan digunakannya server rekonsiliasi penerimaan ini, karena hanya digunakan 1 (satu) jam setiap harinya untuk melakukan \textit{entry} data pembayaran secara otomatis ke server basis data SISMIOP, maka masih banyak \textit{resource} dari server ini yang belum dimanfaatkan secara optimal, sehingga masih dapat digunakan sebagai server 

  \item Personal Komputer
  
Personal komputer diperlukan untuk mengakses server web aplikasi. Tentunya personal komputer yang dapat mengakses server web aplikasi adalah yang terhubung dengan jaringan baik menggunakan kabel jaringan maupun nirkabel. 

Aplikasi web ini pun nantinya dapat diakses melalui jaringan internet melalui personal komputer diluar jaringan lokal Dinas Pendapatan dan Pengelolaan Keuangan. Tujuan dibukanya akses ini ke internet tentu dengan beberapa pembatasan informasi seperti hanya dapat melihat informasi apakah berkas pengajuan pelayanan sudah dapat diambil atau belum, atau informasi yang menyatakan bahwa berkas pengajuan pelayanan BPHTB sedang dalam kondisi ditunda karena beberapa alasan.

Personal komputer yang digunakan untuk tujuan \textit{entry} data ini tidak perlu membeli karena layanan sebelumnya (pelayanan dan pencatatan administrasi manual) sudah menggunakan personal komputer dengan aplikasi Microsoft Office, hanya saja pada saat menggunakan sistem informasi yang akan dibangun ini, nantinya personal komputer yang ada cukup menggunakan \textit{browser} untuk dapat mengakses ke server web aplikasi.

  \item Kartu Jaringan
  
Kartu jaringan digunakan atau dipasangkan pada personal komputer agar personal komputer dapat terhubung dengan jaringan. Tentunya salah satu fungsi dari kabel jaringan ini adalah agar personal komputer yang ada dapat terhubung dengan server aplikasi web.

Kartu ini tidak perlu membeli karena sudah terpasang sebelumnya ditiap personal komputer yang ada, entah sebagai \textit{file sharing} atau sebelumnya pernah melakukan akses SISMIOP atau digunakan untuk mengakses internet.

  \item Kabel Unshielded Twisted Pair (UTP)
  
Kabel ini terdiri dari 4 (empat) pasang warna yang digunakan sebagai penghubung atau media komunikasi antar kartu jaringan yang terdapat pada tiap personal komputer agar personal komputer dapat saling mengakses data.

Kabel ini juga tidak perlu membeli karena dari sistem yang terpasang sebelumnya (pada saat pencatatan administrasi BPHTB secara manual), kabel ini pun sudah terpasang dan saling menghubungkan antara 1 (satu) personal komputer dengan personal komputer yang lain, dengan server, atau dengan printer.

  \item Konektor RJ45
  
Konektor ini digunakan atau dipasangkan pada kabel UTP agar kabel UTP dapat dipasangkan pada kartu jaringan di tiap personal komputer. Masing-masing helai kabel UTP akan terpasang pada tiap pin yang ada pada konektor RJ45, yang kemudian apabila konektor RJ45 dipasangkan pada kartu jaringan, masing-masing pin pada kartu jaringan akan terhubung dengan pin pada RJ45 sehingga aliran data dapat disampaikan dari 1 (satu) personal komputer ke personal komputer yang lain. Dengan kata lain, tiap personal komputer akan saling terhubung.

  \item \textit{Switch}
  
Masing-masing komponen yang terkoneksi / terhubung ke jaringan biasa diistilahkan sebagai \textit{host}, tentunya untuk membuat agar 1 (satu) \textit{host} terhubung dengan 1 (satu) \textit{host} yang lain maka diperlukan sebuah kartu di masing-masing \textit{host} dan sebuah kabel jaringan yang sudah terpasang konektor RJ45. Akan menjadi masalah apabila ada lebih dari 2 (dua) \textit{host} yang perlu terhubung. Rumusnya akan menjadi demikian :

\begin{enumerate}[a.]
  \item Untuk mengetahui jumlah kebutuhan kartu jaringan yaitu menggunakan rumus :

\begin{equation}
  jumlah kartu jaringan = n(n-1)
\end{equation}

dimana n adalah jumlah \textit{host}
  
  \item Sedangkan untuk mengetahui jumlah kabel jaringan yaitu menggunakan rumus :
  
\begin{equation}
  jumlah kabel jaringan = \frac{n(n-1)}{2}
\end{equation}
  
dimana n adalah jumlah \textit{host}
\end{enumerate}

Ini akan lebih efektif dan efisien apabila digunakannya \textit{switch} sebagai penghubung atau pengatur lalu lintas data antar \textit{host}. Sebagai contoh bila ada 2 (dua) \textit{host} yang akan dihubungkan dengan jaringan menggunakan \textit{switch} ini, maka diperlukan 1 (satu) buah \textit{switch}, 2 (dua) buah kabel jaringan (UTP), dan 2 (dua) buah kartu jaringan. Bagaimana bila 3 (tiga) atau lebih \textit{host} akan dihubungkan dalam sebuah jaringan yang sama, maka cukup menggunakan 1 (satu) buah \textit{switch}, n buah kabel jaringan (UTP) dan n buah kartu jaringan sebanyak jumlah \textit{host} yang ingin dihubungkan. Tentunya model \textit{switch} perlu disesuaikan jumlah \textit{port} dengan jumlah \textit{host} yang akan dihubungkan.

\textit{Switch} ini selain menjadikan jaringan komputer lebih efektif dan efisien, juga mengatur lalu lintas data secara lebih baik (dibandingkan \textit{hub}). Salah satu kelebihan yang dimiliki \textit{switch} dibanding \textit{hub} adalah data akan disampaikan tepat kepada alamat yang dituju, sehingga pertukaran data nyaris aman dalam jalur komunikasinya, sedangkan \textit{hub} lebih rentan karena setiap data yang terkirim dan dilewatkan melalui \textit{hub} maka alat ini akan menyebarkan berita ke setiap \textit{host} yang terhubung dengan \textit{hub} yang kemudian diserahkan kepada kartu jaringan apakah data yang diterima sesuai dengan alamat yang dimiliki atau tidak. Dalam kondisi ini, keamanan data akan terlalu berresiko saat terjadi komunikasi.

\textit{Switch} ini pun tidak perlu membeli karena sudah terpasang pada sistem yang lama sebagai penghubung antar 1 (satu) \textit{host} dengan \textit{host} yang lain.

  \item \textit{Router}
  
Alat ini dalam istilah lain juga disebut \textit{gateway}, yaitu sebuah jembatan yang menghubungkan 2 (dua) atau lebih jaringan yang berbeda, cara kerja alat ini yaitu dengan memiliki peta \textit{routing} yang diprogram, peta inilah yang dibaca oleh \textit{router} kemana paket data yang diterima harus dikirimkan, apabila tujuan paket data hanya dikirimkan untuk \textit{host} yang berada di jaringan lokal, maka \textit{router} tidak akan mengirimkan paket datanya keluar, melainkan akan dikirim ke alamat yang berada di dalam, tetapi bila tujuan paket data berada di luar jaringan, maka paket tersebut akan dikirimkan ke jaringan lain dalam lingkupnya, atau diteruskan ke \textit{router} lain yang menghubungkan jaringan diluar lingkupnya. 

Tentunya cara kerja tersebut diatas dengan konfigurasi peta jaringan yang sebelumnya sudah diatur dalam \textit{router}. Dan biasanya, pada setiap \textit{router} terdapat \textit{firewall} yang akan memberikan proteksi atau aturan-aturan komunikasi, dimana tidak semua data dapat diteruskan ke jaringan lain diluarnya, tetapi harus memenuhi persyaratan yang ditetapkan dalam tabel \textit{firewall}.

Jaringan internal Bidang PBB dan BPHTB terdiri dari 2 (dua) jaringan lokal, yaitu jaringan lokal untuk nirkabel di alamat 192.168.3.0/24 dan jaringan lokal dengan kabel (UTP) di alamat 192.168.2.0/24, komunikasi antar kedua jaringan lokal ini dijembatani oleh sebuah \textit{router} yang juga memberikan filter terhadap permintaan data yang masuk ke server. Sebagai contoh, \textit{router} ini pula yang meneruskan data dari internet yang akan mengakses ke server web aplikasi milik bidang PBB dan BPHTB.

Sehingga, biaya untuk pengadaan \textit{router} sendiri tidak diperlukan, karena sudah terpasang dan sudah digunakan untuk kegiatan komunikasi data pada jaringan yang sebelumnya.

  \item \textit{Modem}
  
Fungsi dari \textit{modem} yaitu untuk menghubungkan 2 (dua) atau lebih media jaringan yang berbeda sehingga data dapat dikomunikasikan pada jaringan yang berbeda media tersebut. Sebagai contoh, karena bidang PBB dan BPHTB menggunakan akses \textit{fiber optic} dari jaringan internet, dan jaringan internal menggunakan kabel UTP, maka dibutuhkan \textit{modem} untuk menerjemahkan data yang melintas melalui 2 (dua) media jaringan ini.

Karena \textit{modem} ini sudah digunakan pada sistem jaringan sebelumnya, maka tidak perlu dibeli.

\end{enumerate}.

Sehingga kebutuhan perangkat keras secara keseluruhan sebetulnya tidak perlu mengeluarkan biaya tambahan lain karena perangkat-perangkat tersebut sudah tersedia dan cukup untuk mengakomodir berjalannya sistem informasi yang nantinya akan dibangun.

Kemudian, untuk sarana dan prasarana jenis perangkat lunak yang dibutuhkan yaitu :

\begin{enumerate}[1.]
  \item Perangkat Lunak Sistem Operasi
  
Perangkat lunak sistem operasi digunakan sebagai dasar dari seluruh sistem yang akan berjalan diatasnya. Pemilihan sistem operasi ini pun harus dapat memenuhi kriteria stabil dalam melakukan tugasnya sebagai server untuk rentang waktu 24 jam selama 7 hari berturut-turut.

Tentunya sistem operasi ini sudah terpasang baik pada server-server yang nantinya akan menjadi server basis data dan server aplikasi web. Kedua server ini terpasang sistem operasi CentOS versi 6.3 yang tentu saja gratis sehingga tidak perlu mengeluarkan biaya tambahan baik untuk lisensinya maupun pemasangannya.

  \item Perangkat Lunak Basis Data
  
Perangkat lunak basis data digunakan sebagai sistem yang nantinya akan mengatur dan menyimpan data-data hasil perekaman yang dilakukan oleh petugas pelayanan, yang kemudian hasil perekaman ini dapat diolah untuk menyajikan berbagai macam laporan yang nantinya diproduksi dan dibutuhkan oleh aplikasi web.

Pemilihan perangkat lunak basis data ini pun menggunakan sistem basis data yang gratis namun dengan kualitas yang tidak kalah dengan sistem basis data berbayar. Sehingga tidak perlu mengeluarkan biaya kembali untuk pengadaan perangkat lunak basis data ini. 

Pilihan jatuh kepada perangkat lunak basis data Postgresql, yang tentu saja gratis, tetapi memiliki banyak fasilitas yang berguna seperti membangun \textit{procedure language} yang dapat disesuaikan dengan bahasa pemrograman yang biasa digunakan, memiliki fasilitas pengamanan data \textit{standby database} seperti yang dimiliki oleh basis data Oracle versi Enterprise, dan banyak fasilitas lainnya yang tentu dapat dimanfaatkan dengan lebih optimal sebagaimana jenis perangkat lunak basis data berbayar dengan tanpa mengeluarkan biaya tambahan.
  
  \item Perangkat Lunak Web Server
  
Perangkat lunak \textit{web server} digunakan untuk menjalankan atau menyediakan aplikasi web yang telah dibuat yaitu sistem informasi BPHTB, perangkat lunak ini berbeda dengan \textit{web server} yang digunakan untuk aplikasi web berbasis bahasa pemrograman PHP. Perangkat lunak \textit{web server} yang digunakan harus mendukung \textit{servlet} atau \textit{web container}.

Alasan kenapa dipilih server yang mendukung \textit{servlet} atau \textit{web container} karena memiliki beberapa kelebihan seperti berikut ini :

    \begin{enumerate}[a.]
      \item Efisien dan baik dalam \textit{performance}
      
      \textit{Performance servlet} dapat dikatakan efisien dan baik karena tidak ada proses pembuatan berulang untuk tiap \textit{request} dari \textit{client}. Setiap \textit{request} ditangani oleh proses \textit{servlet container}. \textit{Servlet} tidak dibuat dan dihancurkan secara berulang-ulang, melainkan tetap tersimpan pada memori untuk menangani \textit{request} yang datang selanjutnya.
      
      \item \textit{Powerful}
      
      \textit{Servlet} memiliki kemampuan yang lengkap antara lain mampu melakukan penanganan \textit{request}, \textit{session}, \textit{cookie}, akses ke basis data dengan JDBC dan \textit{caching}, serta pustaka yang lengkap untuk pembuatan aplikasi web.
      
      \item Aman
      
      \textit{Servlet} memiliki fasilitas \textit{security} yang baik dan merupakan bagian dari teknologi Java yang sudah dari asalnya didesain dengan \textit{security} yang baik.
      
      \item Portabilitas
      
      Teknologi Java \textit{servlet} dapat dijalankan di berbagai \textit{servlet container}, \textit{application server}, maupun sistem operasi.
      
      \item Proses \textit{development} yang lebih cepat
      
      Dengan menggunakan \textit{servlet}, kita dapat menggunakan pustaka Java yang lengkap dan menggunakan komponen yang sudah ada tanpa membangun dari awal kembali.
      
      \item Tangguh
      
      \textit{Servlet} merupakan teknologi Java yang memiliki penanganan \textit{memory} yang baik dan \textit{garbage collector} sehingga menjadi aplikasi web yang tangguh dan stabil.
      
      \item Telah digunakan dan diakui di dunia
      
      \textit{Servlet} merupakan teknologi Java yang telah digunakan di berbagai belahan dunia. Dapat ditemukan berbagai komponen, solusi, dan dukungan yang ditawarkan baik secara gratis maupun komersial.
      
      \item Murah
      
      Dikatakan murah karena JDK Java dapat diunduh secara gratis, begitupun dengan \textit{servlet container}.
    \end{enumerate}

Pemilihan perangkat lunak \textit{web server} yang mendukung \textit{servlet} ini pun tidak perlu mengeluarkan biaya, cukup menggunakan Apache Tomcat yang memberikan lisensi gratis dalam penggunaannya, dengan konfigurasi yang tepat, maka \textit{web server} Apache Tomcat ini cukup untuk melayani permintaan aplikasi dari beberapa \textit{client} sekaligus. Sehingga untuk pengadaan perangkat lunak \textit{web server} ini pun tidak memerlukan tambahan biaya.
  
  \item Perangkat Lunak Desain Aplikasi
  
Perangkat lunak desain aplikasi digunakan untuk membuat desain pembangunan aplikasi, mulai dari desain struktur basis data, desain Unified Modeling Language (UML) yang digunakan untuk membuat spesifikasi standar untuk mendokumentasikan, menspesifikan, dan membangun sistem perangkat lunak, desain UML sendiri dapat menggambarkan atau memodelkan diagram struktur aplikasi dan diagram sifat atau cara kerja aplikasi.

Perangkat lunak untuk desain aplikasi ini pun tidak perlu mengeluarkan biaya tambahan, ada perangkat lunak untuk desain aplikasi yang gratis dan disediakan oleh \textit{repository}  Ubuntu yang bernama Dia.

Aplikasi Dia ini memiliki fasilitas bukan hanya untuk membuat desain UML atau struktur basis data, melainkan dapat pula untuk membuat desain jaringan komputer, desain rangkaian elektronik, dan beberapa desain lain sebagaimana terdapat pada halaman \textit{repository} Dia di alamat http://dia-installer.de/shapes/index.html.en.
  
  \item Perangkat Lunak \textit{Integrated Development Environment} (IDE)
  
Perangkat lunak IDE ini digunakan untuk membangun aplikasi web, baik dari sisi tampilan \textit{user interface} maupun dari sisi kode. Pengujian \textit{unit testing} pun dapat dilakukan dengan menggunakan perangkat lunak ini.

Beberapa pilihan untuk perangkat lunak IDE ini pun beragam dan banyak yang menyediakannya secara gratis. Sebagai contoh adalah Netbeans dan Eclipse. Untuk pengembangan aplikasi ini dipilih Eclipse karena tersedianya beragam \textit{plugins} sebagai pendukung dalam membangun sebuah aplikasi.

Dengan kata lain, untuk pengadaan perangkat lunak IDE ini pun tidak perlu mengeluarkan biaya tambahan.
  
  \item Perangkat Lunak Manajemen Basis Data
 
Perangkat lunak manajemen basis data digunakan untuk membangun struktur basis data dengan lebih mudah, biasanya disertai dengan \textit{wizard} yang memandu pengguna untuk membentuk sebuah sistem basis data. Perangkat lunak ini pun dapat digunakan untuk menguji kebenaran kode Structured Query Language (SQL) yaitu kode yang digunakan untuk mengolah data yang berada pada sistem basis data.

Karena basis data yang digunakan adalah Postgresql, maka perangkat lunak untuk manajemen basis data yang digunakan adalah pgAdmin, aplikasi ini pun memiliki lisensi gratis, sehingga tidak perlu lagi mengeluarkan biaya tambahan untuk pengadaannya.
  
  \item \textit{Framework} ZKOSS
  
Karena server yang digunakan aplikasi web yang dibangun berbasis \textit{servlet}, dan bahasa pemrograman yang digunakan \textit{servlet} berbasis Java, maka perlu ditentukan \textit{framework} yang cocok untuk membangun aplikasi dengan lebih cepat dan terstruktur.

\textit{Framework} sendiri merupakan sekumpulan pustaka yang dapat digunakan oleh \textit{developer} untuk mempermudah dalam membangun suatu aplikasi.

\textit{Framework} ZKOSS sebetulnya dalam pengembangan aplikasi web berbasis komponen, persis seperti apa yang ditawarkan oleh IDE untuk membangun sebuah aplikasi \textit{desktop}. Membangun sebuah aplikasi web dengan ZKOSS dipermudah dengan cara menempelkan beberapa komponen seperti \textit{button}, \textit{text area}, dan beberapa \textit{container} untuk menata tampilan, kemudian dihubungkan dengan logika kode program untuk membuatnya bekerja.

\textit{Framework} ZKOSS ini menawarkan beberapa jenis paket diantaranya \textit{Community Edition} yang cukup untuk mengakomodir aplikasi yang akan dibangun, harga untuk jenis paket \textit{Community Edition} ini pun gratis, sehingga tidak memerlukan biaya tambahan untuk pengadaannya.
  
  \item \textit{Framework} Hibernate
  
\textit{Framework} Hibernate ini digunakan sebagai pengatur atau penerjemah dari bahasa Java ke bahasa SQL. Biasa disebut Object Relational Mapping (ORM) karena cara kerjanya adalah dengan membuat kelas-kelas yang merupakan pemetaan (\textit{mapping}) dari tabel-tabel yang terdapat dalam sistem basis data, yang nantinya, pada saat aplikasi akan berhubungan dengan basis data, cukup berurusan dengan kelas-kelas \textit{mapping} ini melalui \textit{framework} Hibernate, kemudian \textit{framework} Hibernate akan menerjemahkan perintah-perintah bahasa pemrograman Java tersebut menjadi SQL untuk diteruskan ke server basis data.

\textit{Framework} Hibernate ini pun gratis untuk digunakan sehingga tidak perlu mengeluarkan biaya lagi untuk pengadaannya, cukup unduh dari alamat http://hibernate.org/org, dan pasangkan pada aplikasi yang akan kita bangun.
  
  \item Pustaka \textit{Driver} Java Database Connection (JDBC)
  
Pustaka \textit{driver} JDBC diperlukan sebagai jembatan antara kode aplikasi yang dibangun dengan data yang berada dalam sistem basis data. \textit{Driver} JDBC ini spesifik dan berbeda untuk tiap sistem basis data, karena sistem basis data yang digunakan adalah Postgresql, maka \textit{driver} yang diperlukan adalah \textit{driver} untuk basis data postgresql yang dapat diunduh di alamat https://jdbc.postgresql.org.

Biasanya tiap-tiap sistem basis data akan menyediakan \textit{driver} JDBC masing-masing sebagai salah satu bentuk dukungan bahwa sistem basis data tersebut mampu ikut terintegrasi bersama aplikasi yang dibangun dengan bahasa pemrograman Java.

Lisensi untuk pustaka \textit{driver} JDBC ini pun gratis sehingga tidak perlu mengeluarkan biaya untuk pengadaannya.

\end{enumerate}

\section{SUMBER DAYA MANUSIA YANG TERLIBAT DALAM PENGOLAHAN DATA}

Sumber daya manusia yang terlibat dalam pengolahan data BPHTB tidak banyak, yaitu petugas yang berfungsi di pelayanan untuk menerima dan memberikan berkas pelayanan ke wajib pajak atau kuasanya, kemudian ada Kepala Seksi Pendataan yang melakukan kontrol berkas, apakah perlu dilakukan pemeriksaan lapangan, atau cukup dilakukan pemeriksaan kantor saja. Kemudian ada Kepala Bidang PBB dan BPHTB yang melakukan tanda tangan dilembar Surat Setoran Pajak Daerah (SSPD) BPHTB sebagai tanda legal bahwa berkas pengajuan tersebut sudah dilakukan verifikasi data tanpa adanya nilai yang kurang bayar.

Bila Kepala Bidang PBB dan BPHTB sedang berhalangan hadir untuk lebih dari 3 (tiga) hari, maka tanda tangan Kepala Dinas akan menggantikan posisi Kepala Bidang PBB dan BPHTB pada lembar SSPD BPHTB sebagai tanda legal bahwa berkas pengajuan tersebut sudah terverifikasi.

Sumber daya manusia lain yang melakukan pengolahan data terhadap pengelolaan BPHTB yaitu petugas yang mencatat penerimaan yang dilaporkan dari bank tempat pembayaran sehingga realisasinya dapat dihitung apakah mencapai target realisasi, atau kurang dari target realisasi, atau melebihi target realisasi.

Ada pula sumber daya manusia yang melakukan kontrol terhadap laporan Pejabat Pembuat Akta Tanah/Pejabat Pembuat Akta Tanah Sementara (PPAT/PPATS), Notaris, dan Kepala Kantor Pelayanan Lelang Negara. Dalam sisi hukumnya, sebagaimana tertuang pada Pasal 92 Undang Undang Nomor 28 Tahun 2009 tentang Pajak Daerah dan Retribusi Daerah, bahwa setiap PPAT/Notaris, dan Kepala Kantor yang membidangi pelayanan lelang negara wajib melaporkan pembuatan akta atau risalah lelang kepada Kepala Daerah dalam hal ini Bupati Brebes paling lambat tanggal 10 (sepuluh) bulan berikutnya, apabila PPAT/Notaris, atau kepala kantor yang membidangi pelayanan lelang negara melanggar Pasal tersebut maka akan dikenakan sanksi administrasi sebagaimana tersebut pada Pasal 93 Undang Undang Nomor 28 Tahun 2009 tentang Pajak Daerah dan Retribusi Daerah.

SDM yang lain yaitu personal yang melakukan cek fisik di lokasi tempat objek BPHTB dialihkan. Hasil dari pengecekan fisik sendiri berupa rekomendasi nilai yang sebenarnya atau nilai pasar yang seharusnya dilaporkan pada lembar SSPD BPHTB.

\section{ORGANISASI SISTEM PENGOLAHAN}

Organisasi sistem pengolahan pada pengelolaan BPHTB cukup sederhana karena memang tujuan pengelolaannya berpangkal pada pencatatan data dan verifikasi data.

Dari sisi paling depan ada bagian yang menjalankan fungsi pelayanan BPHTB, fungsi ini menjalankan tugasnya untuk menerima berkas pelayanan BPHTB dari wajib pajak atau kuasanya dan menyerahkan berkas hasil verifikasi BPHTB kepada wajib pajak atau kuasanya. bagian ini pula yang akan memeriksa kelengkapan berkas pengajuan pelayanan BPHTB, apabila berkas sudah lengkap maka akan dicatatkan dan diteruskan ke bagian berikutnya, namun apabila ada kelengkapan berkas yang kurang maka berkas akan dikembalikan atau ditunda untuk diproses.

Bagian berikutnya adalah Kepala Seksi Pendataan, Penetapan dan Keberatan yang akan melakukan seleksi berkas apakah diperlukan pemeriksaan fisik di lokasi objek berada, atau cukup dilakukan pemeriksaan kantor. Apabila dirasa informasi yang berada pada lembar SSPD BPHTB yang dilaporkan oleh wajib pajak atau kuasanya adalah benar, maka Kepala Seksi Pendataan, Penetapan dan Keberatan akan memberikan paraf pada lembar SSPD BPHTB untuk selanjutnya disampaikan kebagian yang lain. 

Setelah Kepala Seksi Pendataan, Penetapan dan Keberatan meletakkan paraf pada lembar SSPD BPHTB, maka bagian berikutnya yang dituju berkas adalah Kepala Bidang PBB dan BPHTB, Kepala Bidang pun akan melakukan pemeriksaan isian lembar SSPD BPHTB, apabila dirasa benar, maka Kepala Bidang akan memberikan tanda tangan pada lembar tersebut dan berkas diteruskan ke bagian pelayanan, namun apabila ada kejanggalan pada lembar SSPD BPHTB yang diisikan oleh wajib pajak atau kuasanya, maka sebelum berkas diberikan tanda tangan, berkas akan diberikan kepada bagian pemeriksaan lapangan untuk dilakukan pengecekan fisik.
Apabila Kepala Bidang PBB dan BPHTB berhalangan untuk hadir selama atau lebih dari 3 (hari), maka kewenangan pemberian tanda tangan pada lembar SSPD BPHTB dilakukan oleh Kepala Dinas.

Bagian berikutnya adalah petugas pemeriksa lapangan yang melakukan pemeriksaan fisik terhadap objek BPHTB yang dilakukan peralihan. Petugas inilah yang nantinya akan memberikan nilai rekomendasi kepada Kepala Seksi Pendataan, Penetapan, dan Keberatan atau Kepala Bidang PBB dan BPHTB terhadap objek yang diperiksanya. Bila rekomendasi tersebut diambil oleh Kepala Seksi atau Kepala Bidang, maka yang biasanya berlaku adalah bagian pelayanan akan mengeluarkan Surat Tagihan Pajak untuk diberikan kepada wajib pajak atau kuasanya agar melunasi besarnya kurang bayar BPHTB yang dilaporkan kepada bank tempat pembayaran agar berkas tersebut dapat diverifikasi kebenaran datanya.

\section{WAKTU DAN BIAYA YANG DIBUTUHKAN DALAM PEMBUATAN/PENGEMBANGAN SISTEM PENGOLAHAN DATA SECARA MENYELURUH}

Dari sisi biaya, maka tidak diperlukan pengeluaran biaya untuk pembuatan / pengembangan sistem ini karena sebetulnya seluruh perangkat keras sudah terpasang dan berjalan normal, dan dari pengadaan perangkat lunak pun semuanya menggunakan versi gratis.

Sedangkan dari waktu yang dibutuhkan, akan tergambar dari diagram Gantt berikut :



\section{MANFAAT DAN DAMPAK PENGOLAHAN DATA}



\end{document}