%%%%%%%%%%%%%%%%%%%% book.tex %%%%%%%%%%%%%%%%%%%%%%%%%%%%%
%
% sample root file for the chapters of your "monograph"
%
% Use this file as a template for your own input.
%
%%%%%%%%%%%%%%%% Springer-Verlag %%%%%%%%%%%%%%%%%%%%%%%%%%


% RECOMMENDED %%%%%%%%%%%%%%%%%%%%%%%%%%%%%%%%%%%%%%%%%%%%%%%%%%%
\documentclass[pdftex,12pt, oneside]{article}

% choose options for [] as required from the list
% in the Reference Guide, Sect. 2.2
%\usepackage[paperwidth=8.5in, paperheight=13in]{geometry} % Folio
\usepackage[paperwidth=8.27in, paperheight=11.69in]{geometry} % A4

\usepackage{makeidx}         % allows index generation
\usepackage{graphicx}        % standard LaTeX graphics tool
                             % when including figure files
%\usepackage{multicol}        % used for the two-column index
\usepackage[bottom]{footmisc}% places footnotes at page bottom
\usepackage[english]{babel}
\usepackage{enumerate}
\usepackage{paralist}
\usepackage{float}
\usepackage{gensymb}  
\usepackage{listings}
%\usepackage{siunitx}
% etc.
% see the list of further useful packages
% in the Reference Guide, Sects. 2.3, 3.1-3.3
\renewcommand{\baselinestretch}{1.5}

\newcommand{\HRule}{\rule{\linewidth}{0.5mm}}

%\makeindex             % used for the subject index
                       % please use the style svind.ist with
                       % your makeindex program


%%%%%%%%%%%%%%%%%%%%%%%%%%%%%%%%%%%%%%%%%%%%%%%%%%%%%%%%%%%%%%%%%%%%%

\begin{document}
\sloppy % biar section ga melebar melewati kertas
%\author{Priyanto Tamami}
%\title{BUKU PETUNJUK OPERASIONAL SISTEM INFORMASI GEOGRAFIS UNTUK PBB-P2 DENGAN MAPINFO VERSI 8.0}
%\date{22 Desember 2015}
%\maketitle

%\input{./01.title.tex}
\begin{center}
{\large STUDI KELAYAKAN RINCI PENGOLAHAN DATA SISTEM BEA PEROLEHAN HAK ATAS TANAH DAN/ATAU BANGUNAN}
\\[1cm]
16 Maret 2016\\
Priyanto Tamami, S.Kom.
\end{center}

%\frontmatter%%%%%%%%%%%%%%%%%%%%%%%%%%%%%%%%%%%%%%%%%%%%%%%%%%%%%%

%\include{dedic}
%\include{pref}

%\include{02.pengesahan} 

%\tableofcontents
%\listoffigures

%\mainmatter%%%%%%%%%%%%%%%%%%%%%%%%%%%%%%%%%%%%%%%%%%%%%%%%%%%%%%%
%\include{part}
%\include{chapter}
%\include{chapter}
%\appendix
%\include{appendix}

%\include{03.konsep-sig}
%\include{04.pengenalan-software}
%\include{05.koordinat}
%\include{06.registrasi-transformasi-koordinat} 
%\include{07.digitasi-on-screen} 
%\include{08.query} 

%\backmatter%%%%%%%%%%%%%%%%%%%%%%%%%%%%%%%%%%%%%%%%%%%%%%%%%%%%%%%
%\include{solutions}
%\include{referenc}
%\printindex

%%%%%%%%%%%%%%%%%%%%%%%%%%%%%%%%%%%%%%%%%%%%%%%%%%%%%%%%%%%%%%%%%%%%%%

\section{RUANG LINGKUP PEKERJAAN}

Karena luasnya ruang lingkup studi kelayakan tentang pengembangan sistem informasi Bea Perolehan Hak atas Tanah dan/atau Bangunan (BPHTB), dan dengan alasan bahwa dibangunnya sistem informasi sebagai salah satu kebutuhan internal baik sebagai alat untuk pencatatan administrasi yang lebih mudah dan aman, tentunya berkaitan dengan proses pemasukkan data yang mengurangi kesalahan pada sisi \textit{human error}, selain itu dapat pula dilakukan pengembangan-pengembangan fasilitas dari sistem informasi sesuai dengan kebutuhan akan data-data dan informasi yang berkaitan dengan pengelolaan BPHTB, maka beberapa kegiatan yang nantinya dilakukan dalam studi kelayakan ini yaitu :

\begin{enumerate}[1.]
\item Wawancara

Wawancara dilakukan dengan personal yang menangani langsung kegiatan pencatatan administrasi BPHTB, ini dilakukan karena sistem informasi yang akan dibangun nantinya diharapkan dapat mempermudah pekerjaan seorang \textit{data entry} atau personil yang berada di pelayanan dalam mencatatkan berkas pengajuan pelayanan BPHTB. Hal teknis sekecil apapun akan sangat menentukan bagaimana sistem informasi BPHTB nantinya dibangun, baik dalam bentuk data apa saja yang perlu disimpan dalam basis data, tentunya ini akan menentukan bagaimana bentuk tampilan dari aplikasi yang nantinya akan digunakan.

\item Pengumpulan Bahan-Bahan

Pengumpulan bahan-bahan hasil produksi pencatatan yang dilakukan secara manual diperlukan karena ini yang nantinya jadi hasil keluaran bagi sistem informasi BPHTB, sehingga nantinya dapat ditentukan bagaimana sistem informasi akan bekerja untuk mengumpulkan data yang telah dimasukkan dan pada akhirnya akan menghasilkan keluaran yang tidak jauh berbeda dengan apa yang biasa diproduksi pada saat dilakukannya pencatatan administrasi secara manual.
\end{enumerate}

\section{SARANA DAN PRASARANA YANG MELIPUTI PERANGKAT KERAS DAN PERANGKAT LUNAK YANG DIPERLUKAN}




\section{SUMBER DAYA MANUSIA YANG TERLIBAT DALAM PENGOLAHAN DATA}


\section{ORGANISASI SISTEM PENGOLAHAN}


\section{WAKTU DAN BIAYA YANG DIBUTUHKAN DALAM PEMBUATAN/PENGEMBANGAN SISTEM PENGOLAHAN DATA SECARA MENYELURUH}


\section{MANFAAT DAN DAMPAK PENGOLAHAN DATA}


\end{document}