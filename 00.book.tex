%%%%%%%%%%%%%%%%%%%% book.tex %%%%%%%%%%%%%%%%%%%%%%%%%%%%%
%
% sample root file for the chapters of your "monograph"
%
% Use this file as a template for your own input.
%
%%%%%%%%%%%%%%%% Springer-Verlag %%%%%%%%%%%%%%%%%%%%%%%%%%


% RECOMMENDED %%%%%%%%%%%%%%%%%%%%%%%%%%%%%%%%%%%%%%%%%%%%%%%%%%%
\documentclass[pdftex,12pt, oneside]{article}

% choose options for [] as required from the list
% in the Reference Guide, Sect. 2.2
%\usepackage[paperwidth=8.5in, paperheight=13in]{geometry} % Folio
\usepackage[paperwidth=8.27in, paperheight=11.69in]{geometry} % A4

\usepackage{makeidx}         % allows index generation
\usepackage{graphicx}        % standard LaTeX graphics tool
                             % when including figure files
%\usepackage{multicol}        % used for the two-column index
\usepackage[bottom]{footmisc}% places footnotes at page bottom
\usepackage[english]{babel}
\usepackage{enumerate}
\usepackage{paralist}
\usepackage{float}
\usepackage{gensymb}  
\usepackage{listings}
\usepackage{mathtools} % atau \usepackage{amsmath}
%\usepackage{siunitx}
% etc.
% see the list of further useful packages
% in the Reference Guide, Sects. 2.3, 3.1-3.3
\renewcommand{\baselinestretch}{1.5}

\newcommand{\HRule}{\rule{\linewidth}{0.5mm}}

%\makeindex             % used for the subject index
                       % please use the style svind.ist with
                       % your makeindex program


%%%%%%%%%%%%%%%%%%%%%%%%%%%%%%%%%%%%%%%%%%%%%%%%%%%%%%%%%%%%%%%%%%%%%

\begin{document}
\sloppy % biar section ga melebar melewati kertas
%\author{Priyanto Tamami}
%\title{BUKU PETUNJUK OPERASIONAL SISTEM INFORMASI GEOGRAFIS UNTUK PBB-P2 DENGAN MAPINFO VERSI 8.0}
%\date{22 Desember 2015}
%\maketitle

%\input{./01.title.tex}
\begin{center}
{\large STUDI KELAYAKAN RINCI PENGOLAHAN DATA SISTEM BEA PEROLEHAN HAK ATAS TANAH DAN/ATAU BANGUNAN}
\\[1cm]
16 Maret 2016\\
Priyanto Tamami, S.Kom.
\end{center}

%\frontmatter%%%%%%%%%%%%%%%%%%%%%%%%%%%%%%%%%%%%%%%%%%%%%%%%%%%%%%

%\include{dedic}
%\include{pref}

%\include{02.pengesahan} 

%\tableofcontents
%\listoffigures

%\mainmatter%%%%%%%%%%%%%%%%%%%%%%%%%%%%%%%%%%%%%%%%%%%%%%%%%%%%%%%
%\include{part}
%\include{chapter}
%\include{chapter}
%\appendix
%\include{appendix}

%\include{03.konsep-sig}
%\include{04.pengenalan-software}
%\include{05.koordinat}
%\include{06.registrasi-transformasi-koordinat} 
%\include{07.digitasi-on-screen} 
%\include{08.query} 

%\backmatter%%%%%%%%%%%%%%%%%%%%%%%%%%%%%%%%%%%%%%%%%%%%%%%%%%%%%%%
%\include{solutions}
%\include{referenc}
%\printindex

%%%%%%%%%%%%%%%%%%%%%%%%%%%%%%%%%%%%%%%%%%%%%%%%%%%%%%%%%%%%%%%%%%%%%%

\section{RUANG LINGKUP PEKERJAAN}

Karena luasnya ruang lingkup studi kelayakan tentang pengembangan sistem informasi Bea Perolehan Hak atas Tanah dan/atau Bangunan (BPHTB), dan dengan alasan bahwa dibangunnya sistem informasi sebagai salah satu kebutuhan internal baik sebagai alat untuk pencatatan administrasi yang lebih mudah dan aman, tentunya berkaitan dengan proses pemasukkan data yang mengurangi kesalahan pada sisi \textit{human error}, selain itu dapat pula dilakukan pengembangan-pengembangan fasilitas dari sistem informasi sesuai dengan kebutuhan akan data-data dan informasi yang berkaitan dengan pengelolaan BPHTB, maka beberapa kegiatan yang nantinya dilakukan dalam studi kelayakan ini yaitu :

\begin{enumerate}[1.]
\item Wawancara

Wawancara dilakukan dengan personal yang menangani langsung kegiatan pencatatan administrasi BPHTB, ini dilakukan karena sistem informasi yang akan dibangun nantinya diharapkan dapat mempermudah pekerjaan seorang \textit{data entry} atau personil yang berada di pelayanan dalam mencatatkan berkas pengajuan pelayanan BPHTB. Hal teknis sekecil apapun akan sangat menentukan bagaimana sistem informasi BPHTB nantinya dibangun, baik dalam bentuk data apa saja yang perlu disimpan dalam basis data, tentunya ini akan menentukan bagaimana bentuk tampilan dari aplikasi yang nantinya akan digunakan.

\item Pengumpulan Bahan-Bahan

Pengumpulan bahan-bahan hasil produksi pencatatan yang dilakukan secara manual diperlukan karena ini yang nantinya jadi hasil keluaran bagi sistem informasi BPHTB, sehingga nantinya dapat ditentukan bagaimana sistem informasi akan bekerja untuk mengumpulkan data yang telah dimasukkan dan pada akhirnya akan menghasilkan keluaran yang tidak jauh berbeda dengan apa yang biasa diproduksi pada saat dilakukannya pencatatan administrasi secara manual.
\end{enumerate}

\section{SARANA DAN PRASARANA YANG MELIPUTI PERANGKAT KERAS DAN PERANGKAT LUNAK YANG DIPERLUKAN}

Kebutuhan akan sarana dan prasarana untuk jenis perangkat keras yaitu :

\begin{enumerate}[1.]
  \item Server Basis Data
  
Server ini berfungsi untuk menyimpan basis data hasil \textit{entry}, untuk nantinya dapat diakses oleh server aplikasi web (\textit{web app server}) yang dapat dikembangkan bukan hanya sebagai alat untuk \textit{entry} data BPHTB, melainkan dapat menampilkan informasi-informasi lain seperti statistik pelayanan, ataupun kondisi pelaporan lain.

Kebutuhan perangkat keras server ini tidak perlu membeli karena nantinya akan menggunakan server yang sudah ada sebagai server virtualisasi. Karena \textit{resource} simpanan data pada server virtualisasi masih cukup luas dan penggunaannya hanya dilakukan pada saat cetak massal Surat Pemberitahuan Pajak Terhutang (SPPT) Pajak Bumi dan Bangunan Perdesaan dan Perkotaan (PBB-P2).

Jadi server vertualisasi ini akan memiliki beberapa personal komputer virtual didalamnya yang memiliki alamat IP (\textit{Internet Protocol}) mandiri yang dapat diakses dan mengakses jaringan komputer, karena bentuknya adalah personal komputer, maka disana dapat dipasang (\textit{di-install}) aplikasi SISMIOP yang kemudian tentu saja personal komputer virtual ini akan mengakses \textit{printer} untuk melakukan cetak SPPT PBB-P2 di awal tahun.

  \item Server Web Aplikasi
  
Server ini akan digunakan sebagai tempatnya aplikasi web, jadi setiap personal komputer yang akan melakukan \textit{entry} data, atau mengakses informasi dari data-data yang tersimpan di server basis data akan membuka aplikasi yang disediakan oleh \textit{server} ini.
  
Server ini pun tidak perlu membeli karena nantinya akan menggunakan server rekonsiliasi penerimaan PBB-P2. Alasan digunakannya server rekonsiliasi penerimaan ini, karena hanya digunakan 1 (satu) jam setiap harinya untuk melakukan \textit{entry} data pembayaran secara otomatis ke server basis data SISMIOP, maka masih banyak \textit{resource} dari server ini yang belum dimanfaatkan secara optimal, sehingga masih dapat digunakan sebagai server 

  \item Personal Komputer
  
Personal komputer diperlukan untuk mengakses server web aplikasi. Tentunya personal komputer yang dapat mengakses server web aplikasi adalah yang terhubung dengan jaringan baik menggunakan kabel jaringan maupun nirkabel. 

Aplikasi web ini pun nantinya dapat diakses melalui jaringan internet melalui personal komputer diluar jaringan lokal Dinas Pendapatan dan Pengelolaan Keuangan. Tujuan dibukanya akses ini ke internet tentu dengan beberapa pembatasan informasi seperti hanya dapat melihat informasi apakah berkas pengajuan pelayanan sudah dapat diambil atau belum, atau informasi yang menyatakan bahwa berkas pengajuan pelayanan BPHTB sedang dalam kondisi ditunda karena beberapa alasan.

Personal komputer yang digunakan untuk tujuan \textit{entry} data ini tidak perlu membeli karena layanan sebelumnya (pelayanan dan pencatatan administrasi manual) sudah menggunakan personal komputer dengan aplikasi Microsoft Office, hanya saja pada saat menggunakan sistem informasi yang akan dibangun ini, nantinya personal komputer yang ada cukup menggunakan \textit{browser} untuk dapat mengakses ke server web aplikasi.

  \item Kartu Jaringan
  
Kartu jaringan digunakan atau dipasangkan pada personal komputer agar personal komputer dapat terhubung dengan jaringan. Tentunya salah satu fungsi dari kabel jaringan ini adalah agar personal komputer yang ada dapat terhubung dengan server aplikasi web.

Kartu ini tidak perlu membeli karena sudah terpasang sebelumnya ditiap personal komputer yang ada, entah sebagai \textit{file sharing} atau sebelumnya pernah melakukan akses SISMIOP atau digunakan untuk mengakses internet.

  \item Kabel Unshielded Twisted Pair (UTP)
  
Kabel ini terdiri dari 4 (empat) pasang warna yang digunakan sebagai penghubung atau media komunikasi antar kartu jaringan yang terdapat pada tiap personal komputer agar personal komputer dapat saling mengakses data.

Kabel ini juga tidak perlu membeli karena dari sistem yang terpasang sebelumnya (pada saat pencatatan administrasi BPHTB secara manual), kabel ini pun sudah terpasang dan saling menghubungkan antara 1 (satu) personal komputer dengan personal komputer yang lain, dengan server, atau dengan printer.

  \item Konektor RJ45
  
Konektor ini digunakan atau dipasangkan pada kabel UTP agar kabel UTP dapat dipasangkan pada kartu jaringan di tiap personal komputer. Masing-masing helai kabel UTP akan terpasang pada tiap pin yang ada pada konektor RJ45, yang kemudian apabila konektor RJ45 dipasangkan pada kartu jaringan, masing-masing pin pada kartu jaringan akan terhubung dengan pin pada RJ45 sehingga aliran data dapat disampaikan dari 1 (satu) personal komputer ke personal komputer yang lain. Dengan kata lain, tiap personal komputer akan saling terhubung.

  \item \textit{Switch}
  
Masing-masing komponen yang terkoneksi / terhubung ke jaringan biasa diistilahkan sebagai \textit{host}, tentunya untuk membuat agar 1 (satu) \textit{host} terhubung dengan 1 (satu) \textit{host} yang lain maka diperlukan sebuah kartu di masing-masing \textit{host} dan sebuah kabel jaringan yang sudah terpasang konektor RJ45. Akan menjadi masalah apabila ada lebih dari 2 (dua) \textit{host} yang perlu terhubung. Rumusnya akan menjadi demikian :

\begin{enumerate}[a.]
  \item Untuk mengetahui jumlah kebutuhan kartu jaringan yaitu menggunakan rumus :

\begin{equation}
  jumlah kartu jaringan = n(n-1)
\end{equation}

dimana n adalah jumlah \textit{host}
  
  \item Sedangkan untuk mengetahui jumlah kabel jaringan yaitu menggunakan rumus :
  
\begin{equation}
  jumlah kabel jaringan = \frac{n(n-1)}{2}
\end{equation}
  
dimana n adalah jumlah \textit{host}
\end{enumerate}

Ini akan lebih efektif dan efisien apabila digunakannya \textit{switch} sebagai penghubung atau pengatur lalu lintas data antar \textit{host}. Sebagai contoh bila ada 2 (dua) \textit{host} yang akan dihubungkan dengan jaringan menggunakan \textit{switch} ini, maka diperlukan 1 (satu) buah \textit{switch}, 2 (dua) buah kabel jaringan (UTP), dan 2 (dua) buah kartu jaringan. Bagaimana bila 3 (tiga) atau lebih \textit{host} akan dihubungkan dalam sebuah jaringan yang sama, maka cukup menggunakan 1 (satu) buah \textit{switch}, n buah kabel jaringan (UTP) dan n buah kartu jaringan sebanyak jumlah \textit{host} yang ingin dihubungkan. Tentunya model \textit{switch} perlu disesuaikan jumlah \textit{port} dengan jumlah \textit{host} yang akan dihubungkan.

\textit{Switch} ini selain menjadikan jaringan komputer lebih efektif dan efisien, juga mengatur lalu lintas data secara lebih baik (dibandingkan \textit{hub}). Salah satu kelebihan yang dimiliki \textit{switch} dibanding \textit{hub} adalah data akan disampaikan tepat kepada alamat yang dituju, sehingga pertukaran data nyaris aman dalam jalur komunikasinya, sedangkan \textit{hub} lebih rentan karena setiap data yang terkirim dan dilewatkan melalui \textit{hub} maka alat ini akan menyebarkan berita ke setiap \textit{host} yang terhubung dengan \textit{hub} yang kemudian diserahkan kepada kartu jaringan apakah data yang diterima sesuai dengan alamat yang dimiliki atau tidak. Dalam kondisi ini, keamanan data akan terlalu berresiko saat terjadi komunikasi.

\textit{Switch} ini pun tidak perlu membeli karena sudah terpasang pada sistem yang lama sebagai penghubung antar 1 (satu) \textit{host} dengan \textit{host} yang lain.

  \item \textit{Router}
  
Alat ini dalam istilah lain juga disebut \textit{gateway}, yaitu sebuah jembatan yang menghubungkan 2 (dua) atau lebih jaringan yang berbeda, cara kerja alat ini yaitu dengan memiliki peta \textit{routing} yang diprogram, peta inilah yang dibaca oleh \textit{router} kemana paket data yang diterima harus dikirimkan, apabila tujuan paket data hanya dikirimkan untuk \textit{host} yang berada di jaringan lokal, maka \textit{router} tidak akan mengirimkan paket datanya keluar, melainkan akan dikirim ke alamat yang berada di dalam, tetapi bila tujuan paket data berada di luar jaringan, maka paket tersebut akan dikirimkan ke jaringan lain dalam lingkupnya, atau diteruskan ke \textit{router} lain yang menghubungkan jaringan diluar lingkupnya. 

Tentunya cara kerja tersebut diatas dengan konfigurasi peta jaringan yang sebelumnya sudah diatur dalam \textit{router}. Dan biasanya, pada setiap \textit{router} terdapat \textit{firewall} yang akan memberikan proteksi atau aturan-aturan komunikasi, dimana tidak semua data dapat diteruskan ke jaringan lain diluarnya, tetapi harus memenuhi persyaratan yang ditetapkan dalam tabel \textit{firewall}.

Jaringan internal Bidang PBB dan BPHTB terdiri dari 2 (dua) jaringan lokal, yaitu jaringan lokal untuk nirkabel di alamat 192.168.3.0/24 dan jaringan lokal dengan kabel (UTP) di alamat 192.168.2.0/24, komunikasi antar kedua jaringan lokal ini dijembatani oleh sebuah \textit{router} yang juga memberikan filter terhadap permintaan data yang masuk ke server. Sebagai contoh, \textit{router} ini pula yang meneruskan data dari internet yang akan mengakses ke server web aplikasi milik bidang PBB dan BPHTB.

Sehingga, biaya untuk pengadaan \textit{router} sendiri tidak diperlukan, karena sudah terpasang dan sudah digunakan untuk kegiatan komunikasi data pada jaringan yang sebelumnya.

  \item \textit{Modem}
  
Fungsi dari \textit{modem} yaitu untuk menghubungkan 2 (dua) atau lebih media jaringan yang berbeda sehingga data dapat dikomunikasikan pada jaringan yang berbeda media tersebut. Sebagai contoh, karena bidang PBB dan BPHTB menggunakan akses \textit{fiber optic} dari jaringan internet, dan jaringan internal menggunakan kabel UTP, maka dibutuhkan \textit{modem} untuk menerjemahkan data yang melintas melalui 2 (dua) media jaringan ini.

Karena \textit{modem} ini sudah digunakan pada sistem jaringan sebelumnya, maka tidak perlu dibeli.

\end{enumerate}.

Sehingga kebutuhan perangkat keras secara keseluruhan sebetulnya tidak perlu mengeluarkan biaya tambahan lain karena perangkat-perangkat tersebut sudah tersedia dan cukup untuk mengakomodir berjalannya sistem informasi yang nantinya akan dibangun.

Kemudian, untuk sarana dan prasarana jenis perangkat lunak yang dibutuhkan yaitu :

\begin{enumerate}[1.]
  \item Perangkat Lunak Sistem Operasi
  
Perangkat lunak sistem operasi digunakan sebagai dasar dari seluruh sistem yang akan berjalan diatasnya. Pemilihan sistem operasi ini pun harus dapat memenuhi kriteria stabil dalam melakukan tugasnya sebagai server untuk rentang waktu 24 jam selama 7 hari berturut-turut.

Tentunya sistem operasi ini sudah terpasang baik pada server-server yang nantinya akan menjadi server basis data dan server aplikasi web. Kedua server ini terpasang sistem operasi CentOS versi 6.3 yang tentu saja gratis sehingga tidak perlu mengeluarkan biaya tambahan baik untuk lisensinya maupun pemasangannya.

  \item Perangkat Lunak Basis Data
  
Perangkat lunak basis data digunakan sebagai sistem yang nantinya akan mengatur dan menyimpan data-data hasil perekaman yang dilakukan oleh petugas pelayanan, yang kemudian hasil perekaman ini dapat diolah untuk menyajikan berbagai macam laporan yang nantinya diproduksi dan dibutuhkan oleh aplikasi web.

Pemilihan perangkat lunak basis data ini pun menggunakan sistem basis data yang gratis namun dengan kualitas yang tidak kalah dengan sistem basis data berbayar. Sehingga tidak perlu mengeluarkan biaya kembali untuk pengadaan perangkat lunak basis data ini. 

Pilihan jatuh kepada perangkat lunak basis data Postgresql, yang tentu saja gratis, tetapi memiliki banyak fasilitas yang berguna seperti membangun \textit{procedure language} yang dapat disesuaikan dengan bahasa pemrograman yang biasa digunakan, memiliki fasilitas pengamanan data \textit{standby database} seperti yang dimiliki oleh basis data Oracle versi Enterprise, dan banyak fasilitas lainnya yang tentu dapat dimanfaatkan dengan lebih optimal sebagaimana jenis perangkat lunak basis data berbayar dengan tanpa mengeluarkan biaya tambahan.
  
  \item Perangkat Lunak Web Server
  
Perangkat lunak \textit{web server} digunakan untuk menjalankan atau menyediakan aplikasi web yang telah dibuat yaitu sistem informasi BPHTB, perangkat lunak ini berbeda dengan \textit{web server} yang 
  
  \item Perangkat Lunak Desain Aplikasi
  
  \item Perangkat Lunak \textit{Integrated Development Environment} (IDE)
  
  \item Perangkat Lunak Manajemen Basis Data
  
  \item \textit{Framework} ZKOSS
  
  \item \textit{Framework} Hibernate
  
  \item Pustaka \textit{Driver} Java Database Connection (JDBC)
\end{enumerate}

\section{SUMBER DAYA MANUSIA YANG TERLIBAT DALAM PENGOLAHAN DATA}


\section{ORGANISASI SISTEM PENGOLAHAN}


\section{WAKTU DAN BIAYA YANG DIBUTUHKAN DALAM PEMBUATAN/PENGEMBANGAN SISTEM PENGOLAHAN DATA SECARA MENYELURUH}


\section{MANFAAT DAN DAMPAK PENGOLAHAN DATA}


\end{document}