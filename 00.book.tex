%%%%%%%%%%%%%%%%%%%% book.tex %%%%%%%%%%%%%%%%%%%%%%%%%%%%%
%
% sample root file for the chapters of your "monograph"
%
% Use this file as a template for your own input.
%
%%%%%%%%%%%%%%%% Springer-Verlag %%%%%%%%%%%%%%%%%%%%%%%%%%


% RECOMMENDED %%%%%%%%%%%%%%%%%%%%%%%%%%%%%%%%%%%%%%%%%%%%%%%%%%%
\documentclass[pdftex,12pt, oneside]{article}

% choose options for [] as required from the list
% in the Reference Guide, Sect. 2.2
%\usepackage[paperwidth=8.5in, paperheight=13in]{geometry} % Folio
\usepackage[paperwidth=8.27in, paperheight=11.69in]{geometry} % A4

\usepackage{makeidx}         % allows index generation
\usepackage{graphicx}        % standard LaTeX graphics tool
                             % when including figure files
%\usepackage{multicol}        % used for the two-column index
\usepackage[bottom]{footmisc}% places footnotes at page bottom
\usepackage[english]{babel}
\usepackage{enumerate}
\usepackage{paralist}
\usepackage{float}
\usepackage{gensymb}  
\usepackage{listings}
%\usepackage{siunitx}
% etc.
% see the list of further useful packages
% in the Reference Guide, Sects. 2.3, 3.1-3.3
\renewcommand{\baselinestretch}{1.5}

\newcommand{\HRule}{\rule{\linewidth}{0.5mm}}

%\makeindex             % used for the subject index
                       % please use the style svind.ist with
                       % your makeindex program


%%%%%%%%%%%%%%%%%%%%%%%%%%%%%%%%%%%%%%%%%%%%%%%%%%%%%%%%%%%%%%%%%%%%%

\begin{document}

%\author{Priyanto Tamami}
%\title{BUKU PETUNJUK OPERASIONAL SISTEM INFORMASI GEOGRAFIS UNTUK PBB-P2 DENGAN MAPINFO VERSI 8.0}
%\date{22 Desember 2015}
%\maketitle

%\input{./01.title.tex}
\begin{center}
{\large STUDI KELAYAKAN RINCI PENGOLAHAN DATA SISTEM BEA PEROLEHAN HAK ATAS TANAH DAN/ATAU BANGUNAN}
\\[1cm]
16 Maret 2016\\
Priyanto Tamami, S.Kom.
\end{center}

%\frontmatter%%%%%%%%%%%%%%%%%%%%%%%%%%%%%%%%%%%%%%%%%%%%%%%%%%%%%%

%\include{dedic}
%\include{pref}

%\include{02.pengesahan} 

%\tableofcontents
%\listoffigures

%\mainmatter%%%%%%%%%%%%%%%%%%%%%%%%%%%%%%%%%%%%%%%%%%%%%%%%%%%%%%%
%\include{part}
%\include{chapter}
%\include{chapter}
%\appendix
%\include{appendix}

%\include{03.konsep-sig}
%\include{04.pengenalan-software}
%\include{05.koordinat}
%\include{06.registrasi-transformasi-koordinat} 
%\include{07.digitasi-on-screen} 
%\include{08.query} 

%\backmatter%%%%%%%%%%%%%%%%%%%%%%%%%%%%%%%%%%%%%%%%%%%%%%%%%%%%%%%
%\include{solutions}
%\include{referenc}
%\printindex

%%%%%%%%%%%%%%%%%%%%%%%%%%%%%%%%%%%%%%%%%%%%%%%%%%%%%%%%%%%%%%%%%%%%%%

\section{PENENTUAN MASALAH DAN PELUANG YANG DITUJU SISTEM}


\section{PEMBENTUKAN SASARAN SISTEM BARU SECARA KESELURUHAN}


\section{PENGIDENTIFIKASIAN PARA PEMAKAI SISTEM}


\section{PEMBENTUKAN LINGKUP SISTEM}


\section{PENGUSULAN PERANGKAT LUNAK DAN PERANGKAT KERAS UNTUK SISTEM BARU}


\section{PEMBUATAN ANALISIS UNTUK MEMBUAT ATAU MEMBELI APLIKASI}


\section{PEMBUATAN ANALISIS BIAYA/MANFAAT}


\section{PENGKAJIAN TERHADAP RESIKO PROYEK}

\section{PEMBERIAN REKOMENDASI UNTUK MENERUSKAN ATAU MENGHENTIKAN PROYEK}


\end{document}